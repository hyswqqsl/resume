
%%% Local Variables: 
%%% mode: latex
%%% TeX-master: "resume"
%%% End: 

\section{个人简历}

我名叫邵勇,并非学软件出身,酷爱程序设计,程序设计都是自学,从2002年从事软件开
发 以,一直把软件当成一门艺术来做,毎一个程序尽量做到精益求精。 我擅长系统架构
设 计,总体设,并对时间响应快,可靠性高的嵌入式软件有经验。10来年的软件开发过程,
用c,csharp,java开发过许多项目,一直是项目负责人,软件部经理,有项目管理经
验。 带领团队为兰州石化开发了生产安全受控系统(mses),为北京埃昂赛开发了o2o电子交
易平台(cmall),为甘肃移动公司发过移动商务平台; 独立开发了多个光纤光栅测量管理软
件,应用于30多个石油罐区。我的工作经历可以分为四个阶段,2002-2005\ 2005-2008\
2008-2012\ 2012-2014。

\subsection{2002-2005}

工作于国营长风机器厂,这是一家军工企业,有4000名员工,我属于研发部门。在这里主要
作嵌入式开发,用c语言写单片机程序,项目都不大,我一个人负责软件开发,其中有两个
项目值得一提:
\begin{itemize}
\item 便携式数字示波器:应用于军队,产品对显示速度要求很高,用汇编和c混合编程,验
  收时达到满意的精度和速度
\item 风云三号气象卫星伺服软件:与航天部510所合作,从模样、初样、正样,用了三年时
  间,这个项目管理非常严格,对程序和文档反复评审,力求做到产品的完美,自卫星升空
  至今运行正常
\end{itemize}

\subsection{2005-2008}

从长风辞职,就职于兰州奥普信息技术有限公司,是兰州知名软件企业,我担任软件部门
经理,做了很多项目,重要的有:
\begin{itemize}
\item 智能防熘铁鞋:为兰州铁路局开发,产品分发c51开发的下位机,vb开发的上位机,
  之间通过无线方式通讯,系统架构和主要模块由我负责,应用于兰州至天水十余个火车站
\item 移动商务平台:与甘肃移动公司合作,开发移动商务adc服务平台,使用j2ee架
  构,java语言,系统是网络平台,应用于甘肃各地市移动企业用户。

  参与这个项目的有7名软件人员,1名美工。平台应用于甘肃各地市移动企业用户,通过这
  个项目,我学到了如何管理技术团队,与客户沟通,应对各种需求的能加。更重要的是软
  件设计水平有大幅度提高,用敏捷建模和迭代开发模式,使用开源架构,采用单元测试保
  证软件质量,毎个迭代开发三周时间,有完整的设计、开发、测试流程。
\end{itemize}

\subsection{2008-2012}

兰州奥普总部设到了北京,称为北京奥普智信光科,研发搬到北京,位于丰台科技园区。公
司业务转向光纤光栅测量,在北京这段时间在各个方面都有很大的收获。

从开发成果来说,研发成功了各种型号的光纤光栅测量软件,有c51开发的下位机,csharp开
发的上位机,产品应用于30多个石油罐区,电力站。用c51开发中,使用了高焕堂先生介绍
的OOPC方式,使c有了面向对象能力,按照对象UML建模设计,使单片机程序在架构上有质的
飞跃,再加上对象的单元测试,单片机程序的质量也有大的提高。csharp开发中,大量使用
开源软件,由于测量软件对实时性要求高,其中数据库采用开源的sqlite,响应速度快,不
需要安装配置,为上位机成功使用提供了保证。

从技术上说,开始步入自由软件的世界,自2009年,写软件使用emacs,写文档使用
\LaTeX{},绘图使用asymptote,设计使用敏捷建模,写代码的同时编写测试用例。

\subsection{2012-2014}

2012年由于家庭需要,从北京奥普辞职,入职北京埃昂赛甘肃分公司,地址位于兰州石化自动化
研究院,埃昂赛专为公司中国石油研发软件,总部位于北京。在这两年,研发了两个项目。

\subsubsection{mses}

mses全称生产安全受控系统,系统的核心理念是在炼化企业全面推行生产受控管理理念,构
建安全生产受控管理新机制。系统由j2ee开发的服务端,csharp开发移动终端组成,两者通
过webService交互,服务端使用使用springmvc3.0 restful的风格,表现出mvc框架的简单本
质,持久层使用Spring下的JPA模式,数据库使用oracle 10g,使用了大量的存储过程提高数
据库访问速度和可靠性。移动终端的开发中,我使用了反射和元数据技术,实现了一个简
单的orm框架,大大简化了对数据库的访问。

在mses的开发中,我们实践了"测试驱动开发"和"结对编程",效果非常好。测试驱动开发的
方法是一种测试在先,编码在后的开发方法,有效的提高了代码质量。结对编
程(pair-programming)是近年来最为流行的编程方式,两个人写一个程序,其中,一个人
叫driver,另一个人叫observer,driver在编程代码,而observer在旁边实时查看driver的
代码,并帮助driver编程,有效地避免了闭门造车,并可以减少后期的code review时间,以
及代码的学习成本。

\subsubsection{cmall}

cmall电子商务是一个b2c的o2o电子交易平台,平台有多个商户入驻,入驻本网站的商家,在
平台上提供相应的产品和服务,看似类似于传统c2c网站,但有以下不同点:社区设置、顾客
必须要注册登录、商品比价(缺省同一区域内,进入比价页面后可以看全市范围内的)、和论
坛社交紧密结合。系统分为交易平台、商户管理平台和系统管理平台,并计划提供安卓客户
端。我在项目中承担了系统架构的搭建,大量商品数据功能管理,导出数据库、新建数据库、
附加数据等,只需执行一 条指令即可重建数据库,而且一切可配置,实际使用,可以准备多
个数据包(多套csv), 如用于测试的数据包,用于上线的数据包,用于恢复的数据包,一键
可以构建完整数据库。

在cmall的开发中,我们使用了新的项目管理工具maven,比起以往使用ant,简化了很多工
作,maven是基于项目对象模型(pom),来管理项目的编译,测试,构建,报告和文档的软件
项目管理工具和解决依赖关系的工具。


\newpage
