  
%%% Local Variables: 
%%% mode: latex
%%% TeX-master: "resume"
%%% End: 

\section{个人简历}

我名叫邵勇,并非学软件出身,酷爱程序设计,程序设计都是自学,从2002年从事软件开发
以来,一直把软件当成一门艺术来做,毎一个程序尽量做到精益求精。

我擅长软件架构,算法,总体设计,对时间响应快,可靠性高的嵌入式软件有经验。10来年
的软件开发过程,用c,java, c\#开发过许多项目,一直是项目负责人,软件部经理,有
项目管理经验。

我的工作经历可以分为三个阶段,2002至2005年,2005至2008年,2008至2012年。

\subsection{2002年至2005年}

工作于国营长风机器厂,这是一家军工企业,有4000名员工,我属于研发部门。在这里主要
作嵌入式开发,用c语言写单片机程序,项目都不大,我一个人负责软件开发,其中有两个
项目值得一提:
\begin{itemize}
\item 便携式数字示波器:应用于军队,产品对显示速度要求很高,用汇编和c溷合编程,验
  收时达到满意的精度和速度
\item 风云三号气象卫星伺服软件:与航天部510所合作,从模样、初样、正样,用了三年时
  间,这个项目管理非常严格,对程序和文档反复评审,力求做到产品的完美,自卫星升空
  至今运行正常
\end{itemize}

\subsection{2005年至2008年}

从长风辞职,就职于兰州奥普信息技术有限公司,是兰州知名软件企业,我担任软件部门
经理,做了很多项目,重要的有:
\begin{itemize}
\item 智能防熘铁鞋:为兰州铁路局开发,产品分发c51开发的下位机,VB开发的上位机,
  之间通过无线方式通讯,系统架构和主要模块由我负责,应用于兰州至天水十余个火车站
\item 移动商务平台:与甘肃移动公司合作,开发移动商务ADC服务平台,使用J2EE架
  构,java语言,系统是网络平台,应用于甘肃各地市移动企业用户。

  参与这个项目的有7名软件人员,1名美工。平台应用于甘肃各地市移动企业用户,通过这
  个项目,我学到了如何管理技术团队,与客户沟通,应对各种需求的能加。更重要的是软
  件设计水平有大幅度提高,用敏捷建模和迭代开发模式,使用开源架构,采用单元测试保
  证软件质量,毎个迭代开发三周时间,有完整的设计、开发、测试流程。
\end{itemize}

\subsection{2008年至2012年}

兰州奥普总部设到了北京,称为北京奥普智信光科,研发搬到北京,位于丰台科技园区。公
司业务转向光纤光栅测量,在北京这段时间在各个方面都有很大的收获。

从开发成果来说,研发成功了各种型号的光纤光栅测量软件,有c51开发的下位机,csharp开
发的上位机,产品应用于30多个石油罐区,电力站。用c51开发中,使用了高焕堂先生介绍
的OOPC方式,使c有了面向对象能力,按照对象UML建模设计,使单片机程序在架构上有质的
飞跃,再加上对象的单元测试,单片机程序的质量也有大的提高。csharp开发中,大量使用
开源软件,由于测量软件对实时性要求高,其中数据库采用开源的sqlite,响应速度快,不
需要安装配置,为上位机成功使用提供了保证。

从技术上说,开始步入自由软件的世界,自2009年,写软件使用emacs,写文档使用
\LaTeX{},绘图使用asymptote,设计使用敏捷建模,写代码的同时编写测试用例。

\subsection{2013年至2014年}

2012年由于家庭需要,从北京奥普辞职,入职北京埃昂赛甘肃分公司,地址位于兰州石化自动化
研究院,埃昂赛专为公司中国石油研发软件,总部位于北京。在这两年,研发了两个项目。

\subsubsection{mses}

mses全称生产安全受控系统,系统的核心理念是在炼化企业全面推行生产受控管理理念,构
建安全生产受控管理新机制。系统由j2ee开发的服务端,csharp开发移动终端组成,两者通
过webService交互,服务端使用使用springmvc3.0 restful的风格,表现出mvc框架的简单本
质,持久层使用Spring下的JPA模式,数据库使用oracle 10g,使用了大量的存储过程提高数
据库访问速度和可靠性。移动终端的开发中,我们使用了反射和元数据技术,实现了一个简
单的orm框架,大大简化了对数据库的访问。

在mses的开发中,我们实践了"测试驱动开发"和"结对编程",效果非常好。测试驱动开发的
方法是一种测试在先,编码在后的开发方法,有效的提高了代码质量。结对编
程(Pair-Programming)是近年来最为流行的编程方式,所谓结对编程,也就是两个人写一个
程序,其中,一个人叫Driver,另一个人叫Observer,Driver在编程代码,而Observer在旁
边实时查看Driver的代码,并帮助Driver编程。并且,Driver和Observer在一起时可以相互
讨论,有效地避免了闭门造车,并可以减少后期的code review时间,以及代码的学习成本。

\subsubsection{cmall}

cmall电子商务是一个b2c电子交易平台,平台有多个商户入驻,入驻本网站的商家,在
平台上提供相应的产品和服务,看似类似于传统c2c网站,但有以下不同点:
\begin{enumerate}
\item 社区设置,每个门店面对本地区域的顾客(重点,短途物流,配送只支持到本区域,即
  收货人的地址必须在店铺的覆盖区域范围内)
\item 顾客必须要注册登录,登陆后只能看到本地区的门店,可以手动切换区域,顾客注册需
  要填写地区信息
\item 商品比价(缺省同一区域内,进入比价页面后可以看全市范围内的),所有商品都统一
  维护,商品上尽量加条码,有统一条码的可以进行比价,各个门店只是在价格、促销等上面
  有区别,(商户可以选商品加入自己的仓库中,新增的商品要经过审核)
\item 和论坛社交的紧密结合。论坛和社交也要分区域,以区域进行划分,可以展示不同的
  促销公告等。每个区域类似一个地区小论坛和社交圈(群)
\end{enumerate}

系统分为交易平台、商户管理平台和系统管理平台。商户建立时生成一个最高权限的商户管
理员,商户由商户管理员全面负责,可以管理内部门店、内部员工、内部角色,可以从系统
商品中添加商户商品,从商户商品中添加门店商品,每个门店都可以卖商户商品。

系统管理平台有两类人员:系统管理员、客服人员,负责管理整个cmall系统,系统管理员是
系统最高权限,可以管理社区、厂家、商户类别、系统商品,管理每类商户的商品显示类
别,申核商户开户申请,管理客服人员、系统角色。

\newpage
